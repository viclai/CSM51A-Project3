%%%%%%%%%%%%%%%%%%%%%%%%%%%%%%%%%%%%%%%%%
% University/School Laboratory Report
% LaTeX Template
% Version 3.1 (25/3/14)
%
% This template has been downloaded from:
% http://www.LaTeXTemplates.com
%
% Original author:
% Linux and Unix Users Group at Virginia Tech Wiki 
% (https://vtluug.org/wiki/Example_LaTeX_chem_lab_report)
%
% License:
% CC BY-NC-SA 3.0 (http://creativecommons.org/licenses/by-nc-sa/3.0/)
%
%%%%%%%%%%%%%%%%%%%%%%%%%%%%%%%%%%%%%%%%%

%----------------------------------------------------------------------------
%	PACKAGES AND DOCUMENT CONFIGURATIONS
%----------------------------------------------------------------------------

\documentclass{article}

%\usepackage{siunitx}
	% Provides the \SI{}{} and \si{} command for typesetting SI units
\usepackage{graphicx} 
	% Required for the inclusion of images
\graphicspath{ {LaTeX_Images/} }
	% Directory where images are imported from
\usepackage{amsmath} 
	% Required for some math elements 

\setlength\parindent{0pt} 
	% Removes all indentation from paragraphs

%\renewcommand{\labelenumi}{\alph{enumi}.} 
	% Make numbering in the enumerate environment by letter rather than 
	% number (e.g. %section 6)

%\usepackage{times} 
	% Uncomment to use the Times New Roman font

\renewcommand{\vec}[1]{\mathbf{#1}}
%\renewcommand{\textfraction}{0.05}

\usepackage{subfigure}
	% Place figures side by side

%----------------------------------------------------------------------------
%	ABSTRACT
%----------------------------------------------------------------------------

\begin{document}

\begin{abstract}
This is the brief high-level description of the project in plain English text.
\end{abstract}

%----------------------------------------------------------------------------
%	FUNCTIONS OF THE CIRCUIT
%----------------------------------------------------------------------------

\section{Switching Functions of the Circuit}
This section should present both minimal switching expressions in OR-AND form 
and the schematic of the circuit composed of logic gates and flip-flops.

%----------------------------------------------------------------------------
%	VERILOG CODE
%----------------------------------------------------------------------------

\section{Verilog Code}
The following is the implementation of our module in code, pasted from a 
Verilog file. Please also refer to \textit{csm51a\_proj3.v} in the zipped 
file.\\

\begin{verbatim}
Place implementation of code here.
\end{verbatim}

%----------------------------------------------------------------------------
%	SIMULATION RESULT
%----------------------------------------------------------------------------

\section{Simulation Result}
In your simulation, you should show the following 3 sequences: (1) D, N, D;
(2) N, N, N, N, and (3) D, Reset. Please clearly write down the necessary 
information on the waveform diagram so that one of your colleagues who does 
not know anything about your project could understand the behavior of the 
system you are trying to implement.\\

The following test bench code was used to generate that simulation result.

\begin{verbatim}
Place test bench code here.
\end{verbatim}

To observe the output values of each input combination, we changed the input 
every 10 nanoseconds.

%----------------------------------------------------------------------------
%	DESIGN REVIEW
%----------------------------------------------------------------------------

\section{Design Review}
This section includes a summary of your experiences throughout the project. 
It should be no more than TWO pages and may include such topics as what you 
have learned, problem(s) encountered during the implementation and the 
workarounds you came up with, the approach you used, the most important aspects 
of the project for you, where you spent most of the your time and suggestions 
you want to make. In particular, you may include here the new design scheme 
you proposed for the \textit{iKonPlus} controller and alternative 
implementation with MUXes for extra points.

%----------------------------------------------------------------------------
%	TEAM MEMBER CONTRIBUTIONS
%----------------------------------------------------------------------------

\section{Team Member Contributions}
In this section, a detailed description on each team member's responsibility 
and contribution should be presented clearly, including an estimate of 
percentage of efforts on the project and a summary list of each member in the 
project.

%----------------------------------------------------------------------------
%	APPENDIX
%----------------------------------------------------------------------------

\section{Appendix}
This part must show the complete worksheet during the paper-and-paper design.

\subsection{Inputs, Outputs, and States of the System}
Draw the Box!


\subsection{Encoding Schemes}
Give the encoding schemes of the inputs, outputs, and states here.


\subsection{State Diagram and State Table}
Give the state diagram and state table here.


\subsection{Minimization Procedure}
Show the minimization procedure for state and output variables by means of the 
K-map.


\subsection{Final Minimal Expressions}
Show the final minimal expressions of the logic functions in forms of OR-AND 
switching expressions and the final schematic of the circuit.


\subsection{Paper-and-Pencil Design}
The following screenshots show our preliminary work done using paper and 
pencil.

%----------------------------------------------------------------------------
\end{document}